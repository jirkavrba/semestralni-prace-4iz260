\documentclass{article}
\usepackage[utf8]{inputenc}
\usepackage[czech]{babel}
\usepackage{tabularx}
\usepackage{hhline}
\usepackage{url}
\usepackage{xurl}             
\usepackage{hyperref}
\usepackage{biblatex}
\usepackage{bchart}
\usepackage{pgf-pie}  

% Grafy
\usepackage{pgfplots}
\pgfplotsset{compat=1.16}
\usepgfplotslibrary{statistics}

\title{Analýza aplikací na platformě Google Play}
\author{Sergei Naumenko, Jiří Vrba}
\date{4IT260 --- ZS 2021/2022}

\newcolumntype{A}{>{\centering\arraybackslash}X}

\bibliography{sources.bib}

\begin{document}

\maketitle

\begin{minipage}{\dimexpr \linewidth - 15pt \relax}
            \centering
\includegraphics[width=130pt]{FIS_2_logo_cb_rgb.png}
\end{minipage}

\newpage

\section*{Aplikace platformy Google Play}
\par
Daty, se kterými budeme v rámci semestrální práce pracovat, jsou aplikace z platformy Google Play. Google Play je internetový obchod s aplikacemi, 
integrovaný v mobilním operačním systému Android. Na Google Play si může každý zakoupit vývojářskou
licenci a publikovat své aplikace.
Aplikace mohou být buď zdarma, nebo placené a při publikaci jsou vývojářem zařazeny do jedné z
kategorií, což je výčet hodnot definovaný společností Google, která obchod Google Play provozuje.
Po publikaci si mohou uživatelé obchodu aplikaci stáhnout do 
svého zařízení a následně k aplikaci napsat své hodnocení. Hodnocení se skládá z počtu hvězdiček,
a slovního ohodnocení (viz dále v sekci seznámení se s daty).

Hodnocení je důležité pro určování pořadí ve kterém se aplikace v obchodě zobrazují, to je určeno primárně podle počtu stažení aplikace a následně podle průměrného hodnocení.

\section*{Seznámení s daty}
\par
Data se skládají z dvou datasetů. První se zabývá aplikacemi publikovanými na platformu Google Play, druhý pak speciálně jejich hodnocením.

\subsection*{Aplikace publikované na Google Play}
Prvním datasetem je soubor \verb|googleplaystore.csv|. Obsahuje 10840 záznamů a po vyřazení 
duplicit a chybným záznamů jsou k dispozici informace o 9656 aplikacích publikovaných na platfomě Google Play.
Každý záznam obsahuje následující informace:

\vspace*{1em}
\noindent \begin{tabularx}{\linewidth}{|l|l|X|}
    \hline
    Název sloupce & Datový typ & Popis                                    \\ \hhline {|=|=|=|}
    App            & Text    & Název aplikace                             \\ \hline
    Category       & Text    & Kategorie, do které byla aplikace zařazena \\ \hline
    Rating         & Float   & Průměrné hodnocení aplikace (počet hvězd)  \\ \hline
    Reviews        & Integer & Počet hodnocení dané aplikace              \\ \hline
    Size           & Text    & Velikost aplikace, reprezentováno buď číselně (15M), nebo pomocí textu (např. hodnota ,,Varies with device'')\\ \hline
    Installs       & Integer & Počet stažení dané aplikace                \\ \hline
    Type           & Text    & Zdali je aplikace zdarma nebo placená      \\ \hline
    Price          & Float   & Cena aplikace v dolarech                   \\ \hline
    Content Rating & Text    & Věková skupina, které je aplikace určena   \\ \hline
    Genres         & Text    & Seznam žánrů, do které aplikace spadá      \\ \hline
    Last updated   & Date    & Den poslendí aktualizace aplikace          \\ \hline
    Current Ver    & Text    & Verze aplikace (např. 2.0.0)               \\ \hline
    Android Ver    & Text    & Minimální požadovaná verze OS Android      \\ \hline
\end{tabularx}

\subsubsection*{Poznámky k datům}
\begin{itemize}
    \item Hodnocení (sloupec \textbf{Rating}) může nabývat buď hodnoty ,,NaN'', pokud není k dispozici pro danou aplikaci dostatečné
    množství hodnocení aby bylo relevantní hodnotu zobrazovat. Nebo může nabývat desetinných čísel s přesností na 1 desetinné místo 
    od 1.0 do 5.0 včetně.
    
    \item Žánrů může být ve sloupci \textbf{Genres} i více, v takovém případě jsou hodnoty odděleny středníkem.
\end{itemize}

\subsubsection*{Zajímavé poznatky z dat}
\begin{itemize}
    % \item V souboru \textbf{10841} aplikací se nachází pouze \textbf{9656} unikátní záznamů.
    \item Set aplikací je relativně malý vzorek celkového počtu aplikací, kterých je na platformě 
    podle zdroje AppBrain celkem přes 2 780 000 \cite{appbrain_number_of_apps_google_play}
    
    \item Nejpopulárnější kategorií aplikací je kategorie \textbf{FAMILY} (18.19\% instancí), 
        kterou následuje kategorie \textbf{GAME} (10.55\% instancí) a následně kategorie \textbf{TOOLS} 
        (7.78\% instancí)
        
    \begin{bchart}[max=2000, step=200, width=\linewidth]
        \bcbar[text=FAMILY, color=red!50]{1832} \smallskip
        \bcbar[text=GAME, color=red!45]{959} \smallskip
        \bcbar[text=TOOLS, color=red!40]{827} \smallskip
        \bcbar[text=BUSINESS, color=red!35]{420} \smallskip
        \bcbar[text=MEDICAL, color=red!30]{395} \smallskip
        \bcbar[text=PERSONAL..., color=red!25]{375} \smallskip
        \bcbar[text=PRODUCTI..., color=red!20]{374} \smallskip
        \bcbar[text=LIFESTYLE, color=red!15]{369} \smallskip
        \bcbar[text=FINANCE, color=red!10]{345} \smallskip
        \bcbar[text=SPORTS, color=red!5]{325}
        \bcxlabel{Počet aplikací v 10 nejpopulárnějších kategoriích}
    \end{bchart}
        
    \item \textbf{92.62\%} instancí jsou aplikace poskytované zdarma, 
    podle zdroje Statista je v obchodě Google Play celkové zastoupení aplikací poskytovaných zdarma \textbf{96.9\%} \cite{statista_free_paid_apps}
    
    \item \textbf{80.39\%} instancí jsou aplikace určené pro všechny věkové skupiny (\textbf{Everyone})
\end{itemize}


\clearpage

\subsection*{Hodnocení aplikací z platformy Google Play}
\par
Druhým datovým setem je soubor \verb|googleplaystore_user_reviews.csv|. Obsahuje sadu hodnocení aplikací, kde bylo pomocí strojového učení klasifikováno vyznění jako pozitivní, neutrální, pozitivní nebo neurčené.

Tato hodnocení byla sesbírána celkem z \textbf{1074} aplikací a přeložena do angličtiny, před samotnou klasifikací povahy hodnocení.

Celkem se jedná o \textbf{64295} hodnocení, u \textbf{37432} z nich byla klasifikována povaha hodnocení.

Každý záznam obsahuje následující informace:

\vspace*{1em}
\noindent \begin{tabularx}{\linewidth}{|l|l|X|}
    \hline
    Název sloupce & Datový typ & Popis                                    \\ \hhline {|=|=|=|}
    App                & Text    & Název aplikace                             \\ \hline
    Translated Review  & Text    & Text hodnocení přeložený do angličtiny \\ \hline
    Sentiment          & Text    & Pohava hodnocení (pozitivní, negativní, neutrální) \\ \hline
    Sentiment Polarity & Float & Polarita hodnocení (jak moc je hodnocení pozitivní nebo negativní)                \\ \hline
    Sentiment Subjectivity & Float    & Predikovaná míra subjektivity hodnocení \\ \hline
\end{tabularx}

\subsubsection*{Poznámky k datům}
\begin{itemize}
    \item Povaha hodnocení (sloupec \textbf{Sentiment}) může nabývat hodnoty ,,NaN'', pokud model
    pro klasifikaci povahy nebyl schopný určit povahu, a tím pádem tedy i míru subjektivity
    
    \item Polarita hodnocení povahy (sloupec \textbf{Sentiment Polarity}) desetinné je číslo od -1 do 1 udávající konfidenci modelu (1 - model klasifikuje hodnocení jako velice pozitivní, 0 - model klasifikuje hodnocení jako neutrální, -1 - model klasifikuje hodnocení jako negativní)

    \item Subjektivita hodnocení (sloupec \textbf{Sentiment Subjectivity}) je rovněž desetinné číslo od 0 do 1 včentě (1 - hodnocení je podle klasifikace maximálně subjektivní, 0 - hodnocení je podle klasifikace maximálně objektivní)
\end{itemize}

\clearpage

\subsubsection*{Zajímavé poznatky z dat}
\begin{itemize}
    % \item V souboru \textbf{10841} aplikací se nachází pouze \textbf{9656} unikátní záznamů.
    \item U 41.78\% instancí není povaha hodnocení klasifikována, u 37.32\% instancí je povaha klasifikovaná jako pozitivní (Positive), u 12.86\% instancí jako negativní (Negative) a u 8.03\% jako neutrální (Neutral).
        
    \item Negativní a pozitivní hodnocení jsou průměrně více subjektivní než hodnocení klasifikovaná jako neutrální. U negativních hodnocení je průměrná míra subjektivity 0.53, u pozitivních hodnocení je 0.56 a u neurálních hodnocení pouze 0.07.

\end{itemize}

\vspace{50pt}

\begin{minipage}{\linewidth}
            \centering
\begin{tikzpicture}

    \pie[
    color = {
        yellow!30,
        red!30,
        green!30,
        gray!30
        },
        rotate = 90
]{
    8.03/Neutral,
    	12.86/Negative,
    	37.32/Positive,
    	41.78/NaN
    	}
	
\end{tikzpicture}
\\
            \caption{Rozložení klasifikace povahy v datasetu}
        \end{minipage}
\begin{figure}
\centering
\end{figure}


\clearpage

\printbibliography

\end{document}
